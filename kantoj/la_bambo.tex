\begin{intersong}
La Bambo estas tradicia meksika kantaĵo, famega en kaj ekster Esperantujo. Ekzistas ĉefe du Esperantaj versioj de la kanto: unu de Feri Floro, kaj alia de JoMo. Ankaŭ ekzistas Tokipona versio, kiu prezentiĝas sube. Notu ke la akordoj de la versioj malsamas - sed ĉiuj validas por ĉiuj tekstoj.
\end{intersong}

\beginsong{La Bambo}[
	by={Feri Floro},
	cr={tradukis: Feri Floro kaj Gunnar R. FISCHER}
]

\beginverse
Venu dancu la \[C]Bambon \[F] \[G7]
Venu dancu la \[C]Bambon
kaj \[F]kisu \[G]vangon
virinan aŭ \[C]viran \[F] \[G7]
virinan aŭ \[C]viran laŭ \[F]via \[G7]plaĉ'
laŭ via \[C]aĝ', laŭ \[F]via \[G7]aĉ'
\endverse

\beginchorus
\[C]Dan\[F]cu la \[G7]Bambon
\[C]Dan\[F]cu la \[G7]Bambon
\[C]Dan\[F]cu la \[G7]Bambon
\[C]Dan\[F]cu la \[G7]Bambon
\endchorus

\beginverse
Mi ne estas ma^risto ^ ^
Mi ne estas ma^risto
sed ^kapi^tan'
sed kapi^tan'
sed ^kapi^tan'
\endverse

\endsong

\beginsong{La Bambo}[
	by={JoMo},
	cr={tradukis: A. ZOCATO}
]

\beginverse
Por dancadi la \[G]Bambon \[C] \[D]
Por dancadi la \[G]Bambon ne\[C]cesas \[D]jeno:
Kulereto da \[G]ĉarmo \[C] \[D]
Kulereto da \[G]ĉarmo kaj \[C]petol\[D]emo
\textsl{Al la supro, al la \[G]supro \[C] \[D]
Al la supro, al la \[G]supro al\[C]venos \[D]mi
Kaj fariĝos mar\[G]isto \[C] \[D]
Kaj fariĝos mar\[G]isto, se \[C]volas \[D]vi
Se volas \[G]vi, se \[C]volas \[D]vi}
\endverse

\beginchorus
\[G]Dan\[C]cu la \[D]Bambon
\[G]Dan\[C]cu la \[D]Bambon
\[G]Dan\[C]cu la \[D]Bambon
\[G]Dan\[C]cu la \[D]Bambon
\endchorus

\beginverse
Por atingi ĉi^elon ^ ^
Por atingi ĉi^elon ne^cesas ^lerto
Kune kun ŝtupar^ego ^ ^
Kune kun ŝtupar^ego kun ^suple^mento
\textsl{Al la supro, al la supro...}
\endverse

\beginverse
Hejme ĉiuj min ^taksas ^ ^
Hejme ĉiuj min ^taksas na^iva ^viro
Ĉar mi vere ga^lantas ^ ^
Ĉar mi vere ga^lantas al ^junul^inoj
\textsl{Al la supro, al la supro...}
\endverse

\endsong

\beginsong{tawa Bamba}[
	by={La Bambo en Tokipono},
	cr={tradukis: Joop KIEFTE}
]

\beginverse
sina o tawa \[C]musi \[F] \[G7]
sina \[F]o tawa \[C]musi ke\[F]peken \[G7]ijo
sina \[F]o suwi \[C]lili \[F] \[G7]
sina \[F]o suwi \[C]lili \[F]tawa \[G7]musi
\textsl{o \[F]tawa. o \[C]tawa. \[F] \[G7]
o \[F]tawa. mi \[C]tawa lon \[F]poka \[G7]ni
mi o \[F]jan telo \[C]nasa \[F] \[G7]
mi o \[F]jan telo \[C]nasa, tan \[F]sina \[G7]mi
tan \[F]sina \[C]mi, tan \[F]sina \[G7]mi
}
\endverse

\beginchorus
\[C]ta\[F]wa \[G7]Bamba \[F]
\[C]ta\[F]wa \[G7]Bamba \[F]
\[C]ta\[F]wa \[G7]Bamba \[F]
\[C]ta\[F]wa \[G7]Bamba \[F]
\endchorus

\beginverse
mi ken tawa ma ^sewi ^ ^
mi ken ^tawa ma ^sewi ke^peken ^sona
ni la ^wawa li ^wawa ^ ^
ni li ^wawa li ^wawa tan ^ali, li ^pona
\textsl{o tawa...}
\endverse

\beginverse
mi jan pona lon ^tomo ^ ^
mi jan ^pona lon ^tomo. mi ^mije ^pona
mi ken ^pana e ^pona ^ ^
mi ken ^pana e ^pona. a, ^meli li ^kama
\textsl{o tawa...}
\endverse

\endsong





